Corresponding author: leouieda@gmail.com

This is a part of The Leading Edge "Geophysical Tutorials" series.
You can read more about it in \citet{Hall_2016a}.

Open any textbook about seismic data processing and you will inevitably find a
section about the normal moveout (NMO) correction.
There you'll see that we can correct the measured travel-time of a reflected
wave $t$ at a given offset $x$ to obtain the travel-time at normal
incidence $t_0$ by applying the following equation 

\begin{equation}
t_0^2=t^2-\dfrac{x^2}{v_\mathrm{NMO}^2}
\label{eq:traveltime}
\end{equation}

in which $v_\mathrm{NMO}$ is the NMO velocity.
There are variants of this equation with different degrees of accuracy, 
but we'll use this one for simplicity.

When applied to a common midpoint (CMP) section, the equation above is
supposed to turn the hyperbola associated with a reflection into a straight
horizontal line.
What most textbooks won't tell you is \textit{how, exactly, do you apply this
equation to the data}?

Read on and I'll explain step-by-step how the algorithm for NMO correction from
\citet{Yilmaz_2001} works and how to implement it in Python.
The accompanying Jupyter notebook \citep{Perez_2007} contains the full source
code, with documentation and tests for each function.
You can download the notebook at
\href{https://github.com/seg}{github.com/seg} or
find links to run it online at
\href{https://github.com/pinga-lab/nmo-tutorial}{github.com/pinga-lab/nmo-tutorial}.
